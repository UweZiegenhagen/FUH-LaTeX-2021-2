\documentclass[12pt,ngerman,parskip=half]{scrartcl}
\usepackage{babel}
\usepackage{blindtext}
\usepackage{microtype}

\makeatletter % Mach das @ zu einem normal Zeichen
\newcommand{\avg}{\mathop{\operator@font avg}}
\makeatother

\author{Uwe Ziegenhagen}
\title{Mein zweites \LaTeX-Dokument}
\date{Köln, den \today}

\usepackage{hyperref}
\hypersetup{
    bookmarks=true,                     % show bookmarks bar
    unicode=false,                      % non - Latin characters in Acrobat’s bookmarks
    pdftoolbar=true,                        % show Acrobat’s toolbar
    pdfmenubar=true,                        % show Acrobat’s menu
    pdffitwindow=false,                 % window fit to page when opened
    pdfstartview={FitH},                    % fits the width of the page to the window
    pdftitle={My title},                        % title
    pdfauthor={Author},                 % author
    pdfsubject={Subject},                   % subject of the document
    pdfcreator={Creator},                   % creator of the document
    pdfproducer={Producer},             % producer of the document
    pdfkeywords={keyword1, key2, key3},   % list of keywords
    pdfnewwindow=true,                  % links in new window
    colorlinks=true,                        % false: boxed links; true: colored links
    linkcolor=blue,                          % color of internal links
    filecolor=blue,                     % color of file links
    citecolor=blue,                     % color of file links
    urlcolor=blue                        % color of external links
}

\begin{document}

Hallo, ich $a^2+b^2=c^2$ bin eine Formel im Fließtext, in \TeX-Notation.

Hallo, ich \(a^2+b^2=c^2\) bin eine Formel im Fließtext, in \LaTeX-Notation.

Hallo, ich $$a^2+b^2=c^2$$ bin eine abgesetzte Formel, in \TeX-Notation, und sollte nicht mehr genutzt werden.

Hallo, ich \[a^2+b^2=c^2\] bin eine abgesetzte Formel, in \LaTeX-Notation, und sollte immer genutzt werden.

\begin{equation}\label{eq:gl1}% \right. wenn keine schließende große Klammer gesetzt werden soll
-\frac{p}{2} \pm \sqrt{
\left(\frac{p}{2}\right)
^2 - q}
\end{equation}

Siehe Gleichung \ref{eq:gl1} auf Seite \pageref{eq:gl1}.

\begin{equation}
a^{2^3} \Rightarrow \sum_{i=1}^{\infty} i^2 = 4 \not=8 \rightarrow \prod_{i=1}^{\infty} i^2 \quad\mbox{(Siehe dort)}
\end{equation}

% Abstände im Mathemodus:
% \,
% \quad
% \qquad

\begin{equation}
a \cdot b \times c = d \cdots \ddots \ldots \dots \vdots
\end{equation}

\begin{equation}
\overbrace{a^2+b^2}^3 = \underbrace{c^2 + d^2}_4
\end{equation}

\begin{equation}
\sin x \cdot \cos x = \tanh y \avg
\end{equation}


\end{document}

