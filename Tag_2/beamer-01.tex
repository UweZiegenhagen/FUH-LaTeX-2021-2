\documentclass[ngerman]{beamer}
\usepackage{babel}

%\usetheme{Dresden}
\usetheme{PaloAlto}

\author{Uwe Ziegenhagen}
\title{Meine erste Präsentation}
\institute{DANTE e.V. Heidelberg}
\logo{\includegraphics[width=1.5cm]{Bilder/Katze1}}

\begin{document}
\begin{frame}

\maketitle


\end{frame}

\begin{frame}

\tableofcontents

\end{frame}



\section{Einleitung}

\begin{frame}
\frametitle{Einführung}
\framesubtitle{Literatur}

\begin{itemize}
	\item a
	\item b
	\item c
	\item d
	\item e
	\item f
\end{itemize}

\end{frame}

\begin{frame}
\frametitle{Einführung}
\framesubtitle{Literatur}

\scalebox{6}{A}

{\huge
\begin{equation}
-\frac{p}{2} \pm \sqrt{ \left( \frac{p}{2} \right)^2 - q }
\end{equation}}

\end{frame}

\begin{frame}
\frametitle{Einführung}
\framesubtitle{Literatur}

\begin{center}
\includegraphics[width=0.4\textwidth,angle=40]{Bilder/Katze2}
\end{center}

\end{frame}

\begin{frame}
\frametitle{Bla}

\begin{itemize}[<+->]
\item a
\item b
\item c
\item d
\item e
\item f
\end{itemize}
\end{frame}

\begin{frame}
\frametitle{BlaBla}

\begin{itemize}
\item<2-> a
\item<3-> b
\item<-4> c
\item d
\item<5> e
\item f
\end{itemize}
\end{frame}


\end{document}