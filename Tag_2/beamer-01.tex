\documentclass[ngerman,14pt]{beamer}
% Option "handout" ist hilfreich für ebendiese
\usepackage{babel}

%\usetheme{Dresden}
\usetheme{PaloAlto}
% http://www2.informatik.uni-freiburg.de/~frank/latex-kurs/latex-kurs-3/farben/Extra-Farben.pdf
\usecolortheme[named=blue]{structure}

\author{Uwe Ziegenhagen}
\title{Meine erste Präsentation}
\institute{DANTE e.V. Heidelberg}
\logo{\includegraphics[width=1.5cm]{Bilder/Katze1}}

\begin{document}
\begin{frame}

\maketitle


\end{frame}

\begin{frame}

\tableofcontents

\end{frame}



\section{Einleitung}

\begin{frame}
\frametitle{Einführung}
\framesubtitle{Literatur}

\begin{itemize}
	\item a
	\item b
	\item c
	\item d
	\item e
	\item f
\end{itemize}

\end{frame}

\begin{frame}
\frametitle{Einführung}
\framesubtitle{Literatur}

\scalebox{6}{A}

{\huge
\begin{equation}
-\frac{p}{2} \pm \sqrt{ \left( \frac{p}{2} \right)^2 - q }
\end{equation}}

\end{frame}

\begin{frame}
\frametitle{Einführung}
\framesubtitle{Literatur}

\begin{center}
\includegraphics[width=0.4\textwidth,angle=40]{Bilder/Katze2}
\end{center}

\end{frame}

\begin{frame}
\frametitle{Bla}

\begin{itemize}[<+->]
\item a
\item b
\item c
\item d
\item e
\item f
\end{itemize}
\end{frame}

\begin{frame}
\frametitle{BlaBla}

\begin{itemize}
\item<2-> a
\item<3-> b
\item<-4> c
\item d
\item<5> e
\item f
\end{itemize}
\end{frame}

\begin{frame}
\frametitle{Mehrspaltig}

\begin{columns}
\begin{column}{0.48\textwidth}
\begin{itemize}
	\item a
	\item b
	\item c
	\item d
	\item e
	\item f
\end{itemize}
\end{column}
\begin{column}{0.48\textwidth}
\begin{enumerate}
	\item a
	\item b
	\item c
	\item d
	\item e
	\item f
\end{enumerate}
\end{column}

\end{columns}

\end{frame}

\begin{frame}
  \frametitle{Title}
  This is an \alert<+(1)>{example} sentence.
\end{frame}

\setbeamerfont{alerted text}{series=\bfseries}

\begin{frame}
\frametitle{Inhaltsverzeichnis}

\begin{itemize}
\item \alert<1>{example}
\item \alert<2>{example}
\item \alert<3>{example}
\item \alert<4>{example}
\item \alert<5>{example}
\item \alert<6>{example}
\end{itemize}
\end{frame}


\end{document}