\documentclass[12pt,ngerman,parskip=full]{scrreprt}

\usepackage[utf8]{inputenc}
\usepackage[T1]{fontenc}

\usepackage{babel}
\usepackage{blindtext}
\usepackage{microtype}
\usepackage{xcolor}
\usepackage{csquotes}
\usepackage{paralist}

\author{Uwe Ziegenhagen}
\title{Mein erstes \LaTeX-Dokument}
\date{Köln, den 19.11.2021}

\usepackage{hyperref}
\hypersetup{
    bookmarks=true,                     % show bookmarks bar
    unicode=false,                      % non - Latin characters in Acrobat’s bookmarks
    pdftoolbar=true,                        % show Acrobat’s toolbar
    pdfmenubar=true,                        % show Acrobat’s menu
    pdffitwindow=false,                 % window fit to page when opened
    pdfstartview={FitH},                    % fits the width of the page to the window
    pdftitle={My title},                        % title
    pdfauthor={Author},                 % author
    pdfsubject={Subject},                   % subject of the document
    pdfcreator={Creator},                   % creator of the document
    pdfproducer={Producer},             % producer of the document
    pdfkeywords={keyword1, key2, key3},   % list of keywords
    pdfnewwindow=true,                  % links in new window
    colorlinks=true,                        % false: boxed links; true: colored links
    linkcolor=blue,                          % color of internal links
    filecolor=blue,                     % color of file links
    citecolor=blue,                     % color of file links
    urlcolor=blue                        % color of external links
}

\newcommand{\person}[1]{\textsc{\textcolor{green}{#1}}}

\begin{document}
\maketitle

\tableofcontents

\chapter{Wasser}

\section{Einführung}
\subsection{Literaturüberblick}

Hallo Fernuni Hagen, ich bin ein Teil der Einführung in \LaTeX. \person{Heinrich Hertz}\footnote{Die Einheit Hertz wurde nach ihm benannt.}
Hier kommt der Text rein, den ich gesetzt haben möchte. 
Die ganzen Blindtext-Befehle werfen wir raus. \marginpar{Sharp ist eine japa\-nische Firma}

Schon Leonardo da Vinci sagte: \enquote{Trau nicht allen Informationen, \enquote{Trau nicht allen Informationen, die Du im Internet findest.} die Du im Internet findest.}

Siehe Kapitel \ref{cha:fazit} auf Seite \pageref{cha:fazit}.

{\tiny Hallo Welt, ich bin ein Text.}

{\scriptsize Hallo Welt, ich bin ein Text.}

{\footnotesize Hallo Welt, ich bin ein Text.}

{\small Hallo Welt, ich bin ein Text.}

{\normalsize Hallo Welt, ich bin ein Text.}

{\large Hallo Welt, ich bin ein Text.}

{\Large Hallo Welt, ich bin ein Text.}

{\LARGE Hallo Welt, ich bin ein Text.}

{\huge Hallo Welt, ich bin ein Text.}

{\Huge Hallo Welt, ich bin ein Text.}

Hallo, ich bin ein Text in unterschiedlicher Formatierung, mal bin ich \textbf{fett}, mal \textit{kursiv}, mal \textbf{\textit{fett-kursiv}}, manchmal auch wie aus der \texttt{Schreibmaschine}. 

Uwe \\ Ziegenhagen \\ Köln

\person{Albert Einstein} war ein bedeutender Physiker.

\chapter{Hauptteil}\label{cha:hauptteil}

\begin{enumerate}
	\item \enquote{Hallo Welt}
	\item Fernuni
	\item Hagen
	\item Ich bin
	
	\begin{itemize}
		\item 
		\item 
		\item 
	\end{itemize}
	\item eine 
	\item Aufzählung
\end{enumerate}



\blindtext[2]  

\newpage

\begin{itemize}
	\item \enquote{Hallo Welt}
	\item Fernuni
	\item Hagen
	\item Ich bin
	
	\begin{itemize}
	\item \enquote{Hallo Welt}
	\item Fernuni
	\item Hagen
	\item Ich bin
	
	\begin{itemize}
	\item \enquote{Hallo Welt}
	\item Fernuni
	\item Hagen
	\item Ich bin
	\item eine 
	\item Aufzählung
\end{itemize}

	\item eine 
	\item Aufzählung
\end{itemize}
	
	
	\item eine 
	\item Aufzählung
\end{itemize}

\begin{compactitem}[$\Rightarrow$]
	\item \enquote{Hallo Welt}
	\item Fernuni
	\item Hagen
	\item Ich bin
	\item eine 
	\item Aufzählung
\end{compactitem}

\begin{enumerate}[I]
	\item Hallo 
	\item Fernuni
	\item Hagen
	\item Ich bin
	\item eine 
	\item Aufzählung
\end{enumerate}

fuihds udsh ihsdiuh uihsiduh iusfdhui sduih 

\begin{compactenum}[A)] \setcounter{enumi}{6}
	\item Hallo 
	\item Fernuni
	\item Hagen
	\item Ich bin
	\item eine 
	\item Aufzählung
\end{compactenum}

\begin{description}
\item[Apfel] Leckeres Obst
\item[Birne] Auch leckeres Obst
\item[Karotte] Leckeres Gemüse
\end{description}

\begin{compactdesc}
\item[Apfel] Leckeres Obst
\item[Birne] Auch leckeres Obst
\item[Karotte] Leckeres Gemüse
\end{compactdesc}


\chapter{Fazit}\label{cha:fazit}

\section{HalloWelt}\label{sec:hallo}

ßäüöÄÖÜ

\blindtext[2]  

\blindtext[2]  

\blindtext[2]  

\end{document}
