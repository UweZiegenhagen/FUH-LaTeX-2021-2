\documentclass[12pt,ngerman,parskip=full]{scrreprt}

\usepackage{babel}
\usepackage{blindtext}
\usepackage{microtype}

\author{Uwe Ziegenhagen}
\title{Mein erstes \LaTeX-Dokument}
\date{Köln, den 19.11.2021}

\usepackage{hyperref}
\hypersetup{
    bookmarks=true,                     % show bookmarks bar
    unicode=false,                      % non - Latin characters in Acrobat’s bookmarks
    pdftoolbar=true,                        % show Acrobat’s toolbar
    pdfmenubar=true,                        % show Acrobat’s menu
    pdffitwindow=false,                 % window fit to page when opened
    pdfstartview={FitH},                    % fits the width of the page to the window
    pdftitle={My title},                        % title
    pdfauthor={Author},                 % author
    pdfsubject={Subject},                   % subject of the document
    pdfcreator={Creator},                   % creator of the document
    pdfproducer={Producer},             % producer of the document
    pdfkeywords={keyword1, key2, key3},   % list of keywords
    pdfnewwindow=true,                  % links in new window
    colorlinks=true,                        % false: boxed links; true: colored links
    linkcolor=blue,                          % color of internal links
    filecolor=blue,                     % color of file links
    citecolor=blue,                     % color of file links
    urlcolor=blue                        % color of external links
}
\begin{document}
\maketitle

\tableofcontents

\chapter{Wasser}

\section{Einführung}
\subsection{Literaturüberblick}

Hallo Fernuni Hagen, ich bin ein Teil der Einführung in \LaTeX.
Hier kommt der Text rein, den ich gesetzt haben möchte. 
Die ganzen Blindtext-Befehle werfen wir raus.

Siehe Kapitel \ref{cha:fazit} auf Seite \pageref{cha:fazit}.

{\tiny Hallo Welt, ich bin ein Text.}

{\scriptsize Hallo Welt, ich bin ein Text.}

{\footnotesize Hallo Welt, ich bin ein Text.}

{\small Hallo Welt, ich bin ein Text.}

{\normalsize Hallo Welt, ich bin ein Text.}

{\large Hallo Welt, ich bin ein Text.}

{\Large Hallo Welt, ich bin ein Text.}

{\LARGE Hallo Welt, ich bin ein Text.}

{\huge Hallo Welt, ich bin ein Text.}

{\Huge Hallo Welt, ich bin ein Text.}

Hallo, ich bin ein Text in unterschiedlicher Formatierung, mal bin ich \textbf{fett}, mal \textit{kursiv}, mal \textbf{\textit{fett-kursiv}}, manchmal auch wie aus der \texttt{Schreibmaschine}. \textsc{Albert Einstein} war ein bedeutender Physiker.

\chapter{Hauptteil}\label{cha:hauptteil}

\blindtext[200]  

\blindtext[200]  

\blindtext[200]  

\chapter{Fazit}\label{cha:fazit}

\section{HalloWelt}\label{sec:hallo}

\blindtext[2]  

\blindtext[2]  

\blindtext[2]  

\end{document}
