%!TEX TS-program = Arara
% arara: pdflatex: {shell: yes}
\documentclass[12pt,ngerman,parskip=half,DIV=12]{scrbook}

\usepackage{babel}
\usepackage{blindtext}
\usepackage{palatino}


\usepackage[headsepline=0.5pt,footsepline=0.5pt]{scrlayer-scrpage}
\KOMAoptions{headwidth=1\textwidth,footwidth=1\textwidth}

\pagestyle{scrheadings}
\clearscrheadings 

\clearscrheadings 
\ohead[\headmark]{\headmark}
\ofoot[\pagemark]{\pagemark}
\ihead[]{}
\ifoot[]{}
\chead[]{}
\cfoot[]{}




% wenn die Ränder fest vorgegeben sind, geometry benutzen, schöner ist die Nutzung von DIV und BCOR
%\usepackage[left=2cm,right=4cm,top=2cm,bottom=4cm]{geometry}

\author{Uwe Ziegenhagen}
\title{Meine Bachelor-Arbeit}
\date{Köln, den \today}

\begin{document}
\maketitle

\tableofcontents

\listoffigures

\listoftables

\chapter{Einleitung}

\section{Literaturüberblick}

\blindtext[5]

\blindtext[5]

\blindtext[5]

\blindtext[5]

\blindtext[5]

\chapter{Hauptteil}

\section{Literatur im Okzident}

\blindtext[5]

\blindtext[5]

\blindtext[5]

\blindtext[5]

\blindtext[5]

\end{document}