%!TEX TS-program = Arara
% arara: pdflatex: {shell: yes}
% arara: biber
% arara: pdflatex: {shell: yes}

%Standard bei scrbook ist "twoside", für Abschlussarbeiten besser auf "oneside" umstellen.
\documentclass[12pt,ngerman,oneside,parskip=half,DIV=12]{scrbook}

\usepackage{babel}
\usepackage{blindtext}
\usepackage{palatino}

\usepackage[style=authoryear-ibid,backend=biber]{biblatex}

\addbibresource{MeineLiteratur.bib}

\usepackage[headsepline=0.5pt,footsepline=0.5pt]{scrlayer-scrpage}
\KOMAoptions{headwidth=1\textwidth,footwidth=1\textwidth}

\pagestyle{scrheadings}

\clearscrheadings 
\ohead[]{\headmark}
\ofoot[\pagemark]{\pagemark}
\ihead[]{}
\ifoot[]{}
\chead[]{}
\cfoot[]{}

% Wenn nur ein oder mehrere Kapitel neu übersetzt werden sollen; behält die korrekte Seitenzahl
%\includeonly{Kapitel-02}


% wenn die Ränder fest vorgegeben sind, geometry benutzen, schöner ist die Nutzung von DIV und BCOR
%\usepackage[left=2cm,right=4cm,top=2cm,bottom=4cm]{geometry}

\author{Uwe Ziegenhagen}
\title{Meine Bachelor-Arbeit}
\date{Köln, den \today}

\begin{document}
\maketitle

\tableofcontents

\listoffigures

\listoftables

%!TEX TS-program = Arara
%!TeX root = Bachelorarbeit.tex
\chapter{Einleitung}

\section{Literaturüberblick}

Es wäre schön, wenn ich aus dieser Datei übersetzen kann.

\blindtext[5]

\blindtext[5]

\blindtext[5]

\blindtext[5]

\blindtext[5]

\cite{knuth} und \cite{Voss2017} zeigen in ihren \LaTeX-Büchern, wie man mit dem \TeX-System arbeiten kann. Insbesondere \citeauthor{Voss2017} zeigte in seinem im Jahre \citeyear{Voss2017} erschienenen Buch \citetitle{Voss2017}, wie cool \LaTeX\ so ist.

\blindtext \cite{Voss2017} und \cite{knuth} sind wichtige Werke. Man sieht an \cite{Aybas2021} auch, wie wichtig Physik ist.


%!TEX TS-program = Arara
%!TeX root = Bachelorarbeit.tex
\chapter{Hauptteil}

\section{Literatur im Okzident}

\blindtext[5]

\blindtext[5]

\blindtext[5]

\blindtext[5]

\blindtext[5]

\printbibliography

\end{document}